% \program and \endprogram - TeX macros for typesetting algorithms
% David Eppstein / Columbia University / 23 Oct 1987
%
% This is abstracted from a bunch of other macros, and not tested,
% so there may still be some bugs.  Please tell me if you find any.
%
% Keywords are bold, math is italic with roman digits, and | delimits
% roman text.  Outside the display | delimits short program fragments.
% In non-program math mode | is a vertical bar as usual.
% You are more likely to get good results by only using spaces for
% indentation than by also using tabs.  Examples of use:
%
% \program
% for $foo := 1$ to $20$ do
%     |calculate something based on $i$|;
% \endprogram
%
% Text mentioning variable |$foo$| used as the index of a |for| loop...
%
% Note that in the above line, $foo$ without the vertical bars would be
% incorrect, because normal math mode uses strange inter-letter spacing.

\def\program{\begingroup\obeylines\obeytabs\medskip\narrower\narrower\progmode}
\def\endprogram{\endgroup\medskip}

\newif\ifprogmode\progmodefalse		    % Make cond for program mode
\def\progmode{\progmodetrue\progsemi
  \everymath={\progmath}\everydisplay={\progmath}\bf}

\let\semicolon;                             % Semi in programs always is roman
\def\progsemi{\catcode`;=\active}
{\catcode`;=\active\gdef;{\hbox{\rm\semicolon}}}

% Define what to do on entering program math mode.
% : becomes our smart colon that turns := into left rightarrow.
% spaces are no longer active.  letters are italic, but digits are roman.

\def\progmath{\it			    % Italic for labels in programs
  \catcode`:=\active\catcode` =10\catcode`^^I=10    % Colon special, space not
  \mathcode`0="0030\mathcode`1="0031\mathcode`2="0032\mathcode`3="0033
  \mathcode`4="0034\mathcode`5="0035\mathcode`6="0036\mathcode`7="0037
  \mathcode`8="0038\mathcode`9="0039\relax} % Roman not italic numerals

% Colon in program math checks for := and if so becomes left rightarrow

{\catcode`:=\active\gdef:{\futurelet\next\smartcolon}}   % Make colon active
\def\smartcolon{\ifx=\next\let\next\eatgets\else\let\next\colon\fi\next}
\def\eatgets#1{\leftarrow}		    % Eat equal sign and make rightarrow

% Vertical bars are active, bracketing text when in program displays,
% bracketing program mode when in text, and making abs val signs in math.
% The hskip before the begingroup is so that if a | is the start of a program
% line the paragraph will start outside the group and parskip will get set.

\let\vbar=|\catcode`|=\active		    % Save old vbar, make it active
\newif\ifprogblock \progblockfalse
\def|{\ifmmode \vbar                % vbar in math mode acts like before
      \else\ifprogblock \endgroup   % but otherwise it delimits prog stuff
      \else \hskip 0pt\begingroup   % when starting: hskip to start para
        \progblocktrue              % make the next vbar exit
        \ifprogmode \rm             % if this is display prog, delimits text
        \else \progmode \fi\fi\fi}  % if normal text, delimits program literal

% Set up to treat spaces and tabs literally.
% Spaces at the start of lines will indent the line by exactly 0.5em more,
% but spaces in the middle of a line will be treated normally.

{\obeyspaces
\gdef\obeytabs{\parindent=0pt\everypar={\parindent=0pt}\catcode`\^^I=\active
\obeyspaces\let =\smartspace}\obeytabs\gdef^^I{        }}

\def\smartspace{\ifvmode\advance\parindent by.5em\else\space\fi}

