\documentclass{article}

\usepackage{fullpage,latexsym,picinpar,amsmath,amsfonts,graphicx}

\input{macros.tex}

\begin{document}

% v -- YOUR NAME and SID in the braces
\student{Zachary \c Sells}{861013217}
% v -- YOUR NAME and SID in the braces
% v -- ASSIGNMENT NUMBER in the braces
\assignment{3} 
% v-- DUE DATE in the braces
\duedate{Friday, February 22} 

\centerline{\large \bf CS/MATH111 ASSIGNMENT 3}
\centerline{due Thursday, February 21 (8AM)}

\vskip 0.2in
\noindent{\bf Individual assignment:} Problems 1 and 2.

\noindent{\bf Group assignment:} Problems 1,2 and 3.

\vskip 0.1in

%%%%%%%%%%%%%%%%%%%%%%%%%%%%

\begin{problem}
Let $W_n$ be the number of strings of
length $n$ formed from letters $A$, $B$, $C$, 
that do not contain $AB$ or $CC$.
For example, for $n=3$, all the strings with this
property are:
%
\begin{align*}
	AAA, AAC, ACA, ACB, BAA, BAC, BBA, BBB, 
	\\
	 BBC, BCA, BCB, CAA, CAC, CBA, CBB, CBC,
\end{align*}
%
and thus $W_3 = 16$. (Note that $W_0 = 1$, because the
empty string satisfies the condition.)

\noindent (a) Derive a recurrence relation for
        the numbers $W_n$. Justify it.

\noindent (b) Find the formula for the numbers $W_n$
                by solving this recurrence.
                Show your work.
\end{problem}

\textbf{Problem 1 Solution}
\\*
Using the fact that $W_0 = 1, W_1 = 4,$ and $W_2 = 13$ We try to find a formula using these knowns that both agrees with the initial conditions, and gives us the correct number for any given n.
\\*\\*
The number of words that satisfy the conditions is the cardinality of the set{A,B,C} times the value of $W_(n-1)$ plus the value of $W_(n-2)$ giving us a final recurrence equation of $W_n = 3W_(n-1) + W_(n-2)$. We can verify this equation by plugging in $n = 3$. Using counting techniques, we find that the number of words that meet the conditions is 43. And if find this value by using or recurrence equation, we get: $W_4 = 3*(13) + 4$ which also equals 43. 
\\*\\*
Now we must solve this recurrence relation. 
\\*
The characteristic equation of this relation is given by: $x^2 - 3x -1$
\\*
Solving this gives us roots of: $r_1 = \frac{3+\sqrt{13}}{2}$ and $r_2 = \frac{3-\sqrt{13}}{2}$
\\*
Giving us the general equation of: $W_n = \alpha_1 * (\frac{3+\sqrt{13}}{2})^n + \alpha_2 * (\frac{3-\sqrt{13}}{2})^n$
\\*
Plugging in the initial conditions... $\alpha_1 = \frac{1}{2} + \frac{5}{2*\sqrt{13}}$ and $\alpha_2 = \frac{\sqrt{13}-5}{2*\sqrt{13}}$
\\*\\*
Our final solution is... $W_n = (\frac{1}{2} + \frac{5}{2*\sqrt{13}}) * (\frac{3+\sqrt{13}}{2})^n + (\frac{\sqrt{13}-5}{2*\sqrt{13}}) * (\frac{3-\sqrt{13}}{2})^n$

%%%%%%%%%%%%%%%%%%%%%%%%%%%%

\begin{problem}
Solve the following recurrence equation:
%
\begin{eqnarray*}
        f_n &=& 3f_{n-1} + 15f_{n-2} + 2n+3\\
        f_0 &=& 0 \\
        f_1 &=& 1
\end{eqnarray*}
%
Show your work, step by step: the associated homogeneous equation,
the characteristic polynomial and its
roots, the general solution of the homogeneous
equation, computing a particular solution,
the general solution of the non-homogeneous equation,
using the initial conditions to compute the final solution.
\end{problem}

\textbf{Problem 2 Solution}
To solve this recurrence we will break it up into two separate components(Homogeneous and Non-homogeneous), then add them together to get the final solution.
\\*
$f(n) = 3*f(n-1) + 15*f(n-2)$ is the homogeneous part of the equation, which we will find the characteristic equation of.
\\*
The characteristic equation is $x^2 - 3x - 15$.
\\*
Solving this gives roots of: $\frac{3+\sqrt{69}}{2}$ and $\frac{3-\sqrt{69}}{2}$
\\*
This gives us the general solution of $f(n) = \alpha_1*(\frac{3+\sqrt{69}}{2})^n + \alpha_2 * (\frac{3-\sqrt{69}}{2})^n$
\\*\\*
Now we will find the solution to the non-homogeneous part of the recurrence.
\\*
Lets guess the solution: $\beta_1 * n + \beta_2$
\\*
Plugging back into the original recurrence and simplifying gives the two equations: $33*\beta_1 - 17*\beta_2 - 3 = 0$ and $-17\beta_1 - 2 = 0$
\\*
Solving for $\beta_1$ and $\beta_2$ gives $\frac{-2}{17}$ and $\frac{-117}{289}$, respectively.
\\*
Adding this solution to the homogeneous solution gives: $f(n) = \alpha_1*(\frac{3+\sqrt{69}}{2})^n + \alpha_2 * (\frac{3-\sqrt{69}}{2})^n - \frac{2}{17} * n - \frac{117}{289}$
\\*
Now we must solve for the two alphas using the initial conditions
\\*
This gives $\alpha_1 = \frac{529+117*\sqrt{69}}{578*\sqrt{69}}$ and $\alpha_2 = \frac{351 - 23* \sqrt{69}}{1734}$
\\*\\*
Making our final solution: $f(n) = (\frac{529+117*\sqrt{69}}{578*\sqrt{69}}) *(\frac{3+\sqrt{69}}{2})^n + (\frac{351 - 23* \sqrt{69}}{1734}) * (\frac{3-\sqrt{69}}{2})^n - \frac{2}{17} * n - \frac{117}{289}$

%%%%%%%%%%%%%%%%%%%%%%%%%%%%

\begin{problem}
Solve the following recurrence equation:
%
\begin{eqnarray*}
        t_n &=& 3t_{n-1} + 2t_{n-2}\\
        t_0 &=& 0 \\
        t_1 &=& 2
\end{eqnarray*}
%
Show your work.
\end{problem}

\vskip 0.1in
\paragraph{Submission.}
To submit the homework, you need to upload the pdf file into ilearn by 8AM on Thursday, February 21,
and turn-in a paper copy in class (or slip it under my door, no later than 10AM).

\end{document}

