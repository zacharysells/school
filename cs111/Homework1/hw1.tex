
\documentclass{article}

\usepackage{fullpage,latexsym,picinpar,amsmath,amsfonts}

           

%%%%%%%%%%%%%%%%%%%%%%%%%%%%%%%%%%%%%%%%%%%%%%%%%%%%%%%%%%%%%%%%%%%%%%%%%%%%%%%%%%%
%%%%%%%%%%%  LETTERS 
%%%%%%%%%%%%%%%%%%%%%%%%%%%%%%%%%%%%%%%%%%%%%%%%%%%%%%%%%%%%%%%%%%%%%%%%%%%%%%%%%%%

\newcommand{\barx}{{\bar x}}
\newcommand{\bary}{{\bar y}}
\newcommand{\barz}{{\bar z}}
\newcommand{\bart}{{\bar t}}

\newcommand{\bfP}{{\bf{P}}}

%%%%%%%%%%%%%%%%%%%%%%%%%%%%%%%%%%%%%%%%%%%%%%%%%%%%%%%%%%%%%%%%%%%%%%%%%%%%%%%%%%%
%%%%%%%%%%%%%%%%%%%%%%%%%%%%%%%%%%%%%%%%%%%%%%%%%%%%%%%%%%%%%%%%%%%%%%%%%%%%%%%%%%%
                                                                                
\newcommand{\parend}[1]{{\left( #1  \right) }}
\newcommand{\spparend}[1]{{\left(\, #1  \,\right) }}
\newcommand{\angled}[1]{{\left\langle #1  \right\rangle }}
\newcommand{\brackd}[1]{{\left[ #1  \right] }}
\newcommand{\spbrackd}[1]{{\left[\, #1  \,\right] }}
\newcommand{\braced}[1]{{\left\{ #1  \right\} }}
\newcommand{\leftbraced}[1]{{\left\{ #1  \right. }}
\newcommand{\floor}[1]{{\left\lfloor #1\right\rfloor}}
\newcommand{\ceiling}[1]{{\left\lceil #1\right\rceil}}
\newcommand{\barred}[1]{{\left|#1\right|}}
\newcommand{\doublebarred}[1]{{\left|\left|#1\right|\right|}}
\newcommand{\spaced}[1]{{\, #1\, }}
\newcommand{\suchthat}{{\spaced{|}}}
\newcommand{\numof}{{\sharp}}
\newcommand{\assign}{{\,\leftarrow\,}}
\newcommand{\myaccept}{{\mbox{\tiny accept}}}
\newcommand{\myreject}{{\mbox{\tiny reject}}}
\newcommand{\blanksymbol}{{\sqcup}}
                                                                                                                         
\newcommand{\veps}{{\varepsilon}}
\newcommand{\Sigmastar}{{\Sigma^\ast}}
                           
\newcommand{\half}{\mbox{$\frac{1}{2}$}}    
\newcommand{\threehalfs}{\mbox{$\frac{3}{2}$}}   
\newcommand{\domino}[2]{\left[\frac{#1}{#2}\right]}  

\newcommand{\naturals}{{\mathbb{N}}}

%%%%%%%%%%%%%%%%%%%%%%%%%%%%%%%%%%%%%%%%%%%%%%%%%%%%%%%%%%%%%%%%%%%%%%%%%%%%%%%%%%%
%%%%%%%%%%%%%%% for homeworks
%%%%%%%%%%%%%%%%%%%%%%%%%%%%%%%%%%%%%%%%%%%%%%%%%%%%%%%%%%%%%%%%%%%%%%%%%%%%%%%%%%%

\newcommand{\student}[2]{%
{\noindent\Large{ \emph{#1} SID {#2} } \hfill} \vskip 0.1in}

\newcommand{\assignment}[1]{\medskip\centerline{\large\bf CS 111 ASSIGNMENT {#1}}}

\newcommand{\duedate}[1]{{\centerline{due {#1}\medskip}}}     

\newcounter{problemnumber}                                                                                 

\newenvironment{problem}{{\vskip 0.1in \noindent
              \bf Problem~\addtocounter{problemnumber}{1}\arabic{problemnumber}:}}{}

\newcounter{solutionnumber}

\newenvironment{solution}{{\vskip 0.1in \noindent
             \bf Solution~\addtocounter{solutionnumber}{1}\arabic{solutionnumber}:}}
				{\ \newline\smallskip\lineacross\smallskip}

\newcommand{\lineacross}{\noindent\mbox{}\hrulefill\mbox{}}

\newcommand{\decproblem}[3]{%
\medskip
\noindent
\begin{list}{\hfill}{\setlength{\labelsep}{0in}
                       \setlength{\topsep}{0in}
                       \setlength{\partopsep}{0in}
                       \setlength{\leftmargin}{0in}
                       \setlength{\listparindent}{0in}
                       \setlength{\labelwidth}{0.5in}
                       \setlength{\itemindent}{0in}
                       \setlength{\itemsep}{0in}
                     }
\item{{{\sc{#1}}:}}
                \begin{list}{\hfill}{\setlength{\labelsep}{0.1in}
                       \setlength{\topsep}{0in}
                       \setlength{\partopsep}{0in}
                       \setlength{\leftmargin}{0.5in}
                       \setlength{\labelwidth}{0.5in}
                       \setlength{\listparindent}{0in}
                       \setlength{\itemindent}{0in}
                       \setlength{\itemsep}{0in}
                       }
                \item{{\em Instance:\ }}{#2}
                \item{{\em Query:\ }}{#3}
                \end{list}
\end{list}
\medskip
}

%%%%%%%%%%%%%%%%%%%%%%%%%%%%%%%%%%%%%%%%%%%%%%%%%%%%%%%%%%%%%%%%%%%%%%%%%%%%%%%%%%%
%%%%%%%%%%%%% for quizzes
%%%%%%%%%%%%%%%%%%%%%%%%%%%%%%%%%%%%%%%%%%%%%%%%%%%%%%%%%%%%%%%%%%%%%%%%%%%%%%%%%%%

\newcommand{\quizheader}{ {\large NAME: \hskip 3in SID:\hfill}
                                \newline\lineacross \medskip }


%%%%%%%%%%%%%%%%%%%%%%%%%%%%%%%%%%%%%%%%%%%%%%%%%%%%%%%%%%%%%%%%%%%%%%%%%%%%%%%%%%%
%%%%%%%%%%%%% for final
%%%%%%%%%%%%%%%%%%%%%%%%%%%%%%%%%%%%%%%%%%%%%%%%%%%%%%%%%%%%%%%%%%%%%%%%%%%%%%%%%%%

\newcommand{\namespace}{\noindent{\Large NAME: \hfill SID:\hskip 1.5in\ }\\\medskip\noindent\mbox{}\hrulefill\mbox{}}


%%%%%%%%%%%%%%%%%%%%%%%%%%%%%%%%%%%%%%%%%%%%%%%%%%%%%%%%%%%%%%%%%%%%%%%%%%%%%%%%%%%
%%%%%%%%%%%%% for notes
%%%%%%%%%%%%%%%%%%%%%%%%%%%%%%%%%%%%%%%%%%%%%%%%%%%%%%%%%%%%%%%%%%%%%%%%%%%%%%%%%%%


\newtheorem{theorem}{Theorem}[section]
\newtheorem{definition}[theorem]{Definition}
\newtheorem{corollary}[theorem]{Corollary}
\newtheorem{lemma}[theorem]{Lemma}
\newtheorem{fact}[theorem]{Fact}
\newtheorem{claim}[theorem]{Claim}

\newenvironment{proof}{{\it Proof:\/}}{$\Box$\vskip 0.1in}



\begin{document}

% v -- YOUR NAME and SID in the braces
\student{Zachary \c Sells}{861013217}
% v -- YOUR NAME and SID in the braces
% v -- ASSIGNMENT NUMBER in the braces
\assignment{1} 
% v-- DUE DATE in the braces
\duedate{Thursday, January 24}  

\medskip

%%%%%%%%%%%%%%%%%%%%%%%%%%%%%%%%%%%%%%%%%%%%%%%%%%%%%%%%%%%%%%%%%%%%%%%%%%


\centerline{\large \bf CS/MATH111 ASSIGNMENT 1}
\centerline{due Thursday, January 24 (8AM))}

\vskip 0.1in
\noindent{\bf Individual assignment:} Problems 1 and 2.

\noindent{\bf Group assignment:} Problems 1,2 and 3.

\vskip 0.2in

%%%%%%%%%%%%%%%%%%%%%%%%%%%%

\begin{problem}
(a)
Give the exact formula (as a function of $n$) for the number of
times ``bingo" is printed by Algorithm~\textsc{BingoPrint} below.
First express it as a summation formula and justify it. Then simplify it to 
obtain a closed-form expression. Show your derivation.

\noindent
(b)
Give the asymptotic value of the
number of ``bingo"s using the $\Theta$-notation. Include a brief justification. You will need a formula
for the sum of consecutive squares that you can find on the internet.

\begin{tabbing}
aa \= aa \= aa \= aa \= aa \= aa \= \kill
\textbf{Algorithm} \textsc{BingoPrint} $(n: \mbox{\bf integer})$ \\
      \> \textbf{for} $i \leftarrow 1$ \textbf{to} $2n+1$
                         \textbf{do} \\
      \> \> \textbf{for} $j \leftarrow 1$ \textbf{to} $i^2+2i$ \textbf{do} \\
      \> \> \> print(``bingo")
      \\* \\* \\*
\end{tabbing}
\end{problem}


\textbf{Solution 1:} 

The summation formula for the BINGOPRINT algorithm is: 
$\displaystyle\sum\limits_{j=1}^{2n+1} j^2+2j$
Because the function goes into the first forloop $2n+1$ times. The function then runs through the 2nd forloop $j^2 + 2j$ times for every outer loop iteration.
We know...
$\displaystyle\sum\limits_{j=1}^{2n+1} j^2+2j = \sum\limits_{j=1}^{2n+1} j^2 + \sum\limits_{j=1}^{2n+1} 2j$


So...
$\displaystyle\sum\limits_{j=1}^{2n+1} j^2 = \frac{(2n+1)(2n+2)(2(2n+1)+1)}{6}$
And...
$2 * \displaystyle\sum\limits_{j=1}^{2n+1} j = (2n+1)(2n+2)$
\\*
Which gives us...
$\displaystyle\frac{(4n^2 + 6n + 2)(4n+3)}{6} * (4n^2 + 6n + 2)$
\\* \\*
That simplifies down to....
$\frac{8n^3}{3} + 10n^2 + \frac{31n}{3} + 3$ Which is our closed form expression of the answer.
\\* \\* \\*
We know that $\frac{8n^3}{3} + 10n^2 + \frac{31n}{3} + 3 \geq \frac{8n^3}{3} + 10n^3 = \frac{38n^3}{3}$ for all n $\geq$ 2
\\* \\* \\*
So...
$\frac{8n^3}{3} + 10n^2 + \frac{31n}{3} + 3 = \mathcal{O}(n^3)$ with c = $\frac{38n^3}{3}$
\\* \\* \\*
We know that $\frac{8n^3}{3} + 10n^2 + \frac{31n}{3} + 3 \leq \frac{8n^3}{3}$
\\* \\* \\*
So...
$\frac{8n^3}{3} + 10n^2 + \frac{31n}{3} + 3 = \Omega(n^3)$ with c = $\frac{8n^3}{3}$


Therefore the asymptotic notation is as follows...
$\frac{8n^3}{3} + 10n^2 + \frac{31n}{3} + 3 = \theta(n^3)$
\\* \\* \\* \\*

%%%%%%%%%%%%%%%%%%%%%%%%%%%%

\begin{problem}
Use mathematical induction to prove that $3^n \ge n2^n$ for $n\ge 0$. 
(Note: dealing with the base case may require some thought.)
\\* \\* \\*
\end{problem}

\textbf{Solution 2:}\\

Base Case:\\
For n = 0, $3^0 \geq 0*2^0$ holds true\\
For n = 1, $3^1 \geq 1*2^1$ holds true\\
For n = 2, $3^2 \geq 2*2^2$ holds true\\

Induction Step:\\
Assume for n=k, that: $3^k \geq k*2^k$ is true\\ \\
Now we must prove that $3^{(k+1)} \geq (k+1)*2^{(k+1)}$ is also true\\ \\
$3^{(k+1)} = 3^k + 2*(3^k)$\\ \\
Expanding the right side of the inequality gives us...\\ \\ 
$3^k + 2*(3^k) \geq k*2^{(k+1)} + 2^{(k+1)}$ \\ \\ 
By expanding the "$2^{(k+1)}$" terms...\\ \\ 
$3^k + 2*(3^k) \geq k*2^k * 2 + 2^k * 2$\\ \\ 
Factoring out a "2"....\\ \\ 
$3^k + 2*(3^k) \geq 2(k*2^k + 2^k)$\\ \\ 
We can multiply the right side by "1/2" because any number multiplied by "1/2" will make it smaller, so our inequality will still hold.\\ \\
$3^k + 2*(3^k) \geq k*2^k + 2^k$\\ \\
If both each term on the left is greater than each term on the right, then the sum of the terms on the left will be greater than the sum of the terms on the right, For all n$\geq$0\\ \\
$3^k \geq k*2^k$ Is true because of our inductive assumption\\ \\
$2*(3^k) \geq 2^k$ Is obviously true for all n$\geq$0\\ \\
End of Proof\\ \\ \\ \\






%%%%%%%%%%%%%%%%%%%%%%%%%%%%

\begin{problem}
Give the asymptotic values of the
following functions, using the $\Theta$-notation:
%
\begin{description}
%
\item{(a)} $9n^2 + n^3/2 + 29n + 13$
\item{(b)} $\sqrt{n}+ 7\log^5 n + 2n\log n$
\item{(c)} $1+ n^3\log^3n + 21 n^2\log^4n$
\item{(d)} $\log^7n + n 2^n + 13n\log^9n$
\item{(e)} $n^53^n+4^n$
%
\end{description}
%
Justify your answer.
Here, you don't need to give a complete rigorous proof.
Give only an informal explanation using asymptotic
relations between the functions $n^c$, $\log n$, and $c^n$.
\end{problem}

%%%%%%%%%%%%%%%%%%%%%%%%%%%%

\vskip 0.1in
\paragraph{Submission.}
To submit the homework, you need to upload the pdf file into ilearn by 8AM on Thursday, January 24,
\textbf{and} turn-in a paper copy in class.

\paragraph{Reminders.}
Remember that only {\LaTeX} papers are accepted. Also,
remember to turn in your signed academic integrity statement.
You can scan it and submit via ilearn, turn it in in class, or
slip it under my office door.


\end{document}

