
\documentclass{article}

           

%%%%%%%%%%%%%%%%%%%%%%%%%%%%%%%%%%%%%%%%%%%%%%%%%%%%%%%%%%%%%%%%%%%%%%%%%%%%%%%%%%%
%%%%%%%%%%%  LETTERS 
%%%%%%%%%%%%%%%%%%%%%%%%%%%%%%%%%%%%%%%%%%%%%%%%%%%%%%%%%%%%%%%%%%%%%%%%%%%%%%%%%%%

\newcommand{\barx}{{\bar x}}
\newcommand{\bary}{{\bar y}}
\newcommand{\barz}{{\bar z}}
\newcommand{\bart}{{\bar t}}

\newcommand{\bfP}{{\bf{P}}}

%%%%%%%%%%%%%%%%%%%%%%%%%%%%%%%%%%%%%%%%%%%%%%%%%%%%%%%%%%%%%%%%%%%%%%%%%%%%%%%%%%%
%%%%%%%%%%%%%%%%%%%%%%%%%%%%%%%%%%%%%%%%%%%%%%%%%%%%%%%%%%%%%%%%%%%%%%%%%%%%%%%%%%%
                                                                                
\newcommand{\parend}[1]{{\left( #1  \right) }}
\newcommand{\spparend}[1]{{\left(\, #1  \,\right) }}
\newcommand{\angled}[1]{{\left\langle #1  \right\rangle }}
\newcommand{\brackd}[1]{{\left[ #1  \right] }}
\newcommand{\spbrackd}[1]{{\left[\, #1  \,\right] }}
\newcommand{\braced}[1]{{\left\{ #1  \right\} }}
\newcommand{\leftbraced}[1]{{\left\{ #1  \right. }}
\newcommand{\floor}[1]{{\left\lfloor #1\right\rfloor}}
\newcommand{\ceiling}[1]{{\left\lceil #1\right\rceil}}
\newcommand{\barred}[1]{{\left|#1\right|}}
\newcommand{\doublebarred}[1]{{\left|\left|#1\right|\right|}}
\newcommand{\spaced}[1]{{\, #1\, }}
\newcommand{\suchthat}{{\spaced{|}}}
\newcommand{\numof}{{\sharp}}
\newcommand{\assign}{{\,\leftarrow\,}}
\newcommand{\myaccept}{{\mbox{\tiny accept}}}
\newcommand{\myreject}{{\mbox{\tiny reject}}}
\newcommand{\blanksymbol}{{\sqcup}}
                                                                                                                         
\newcommand{\veps}{{\varepsilon}}
\newcommand{\Sigmastar}{{\Sigma^\ast}}
                           
\newcommand{\half}{\mbox{$\frac{1}{2}$}}    
\newcommand{\threehalfs}{\mbox{$\frac{3}{2}$}}   
\newcommand{\domino}[2]{\left[\frac{#1}{#2}\right]}  

\newcommand{\naturals}{{\mathbb{N}}}

%%%%%%%%%%%%%%%%%%%%%%%%%%%%%%%%%%%%%%%%%%%%%%%%%%%%%%%%%%%%%%%%%%%%%%%%%%%%%%%%%%%
%%%%%%%%%%%%%%% for homeworks
%%%%%%%%%%%%%%%%%%%%%%%%%%%%%%%%%%%%%%%%%%%%%%%%%%%%%%%%%%%%%%%%%%%%%%%%%%%%%%%%%%%

\newcommand{\student}[2]{%
{\noindent\Large{ \emph{#1} SID {#2} } \hfill} \vskip 0.1in}

\newcommand{\assignment}[1]{\medskip\centerline{\large\bf CS 111 ASSIGNMENT {#1}}}

\newcommand{\duedate}[1]{{\centerline{due {#1}\medskip}}}     

\newcounter{problemnumber}                                                                                 

\newenvironment{problem}{{\vskip 0.1in \noindent
              \bf Problem~\addtocounter{problemnumber}{1}\arabic{problemnumber}:}}{}

\newcounter{solutionnumber}

\newenvironment{solution}{{\vskip 0.1in \noindent
             \bf Solution~\addtocounter{solutionnumber}{1}\arabic{solutionnumber}:}}
				{\ \newline\smallskip\lineacross\smallskip}

\newcommand{\lineacross}{\noindent\mbox{}\hrulefill\mbox{}}

\newcommand{\decproblem}[3]{%
\medskip
\noindent
\begin{list}{\hfill}{\setlength{\labelsep}{0in}
                       \setlength{\topsep}{0in}
                       \setlength{\partopsep}{0in}
                       \setlength{\leftmargin}{0in}
                       \setlength{\listparindent}{0in}
                       \setlength{\labelwidth}{0.5in}
                       \setlength{\itemindent}{0in}
                       \setlength{\itemsep}{0in}
                     }
\item{{{\sc{#1}}:}}
                \begin{list}{\hfill}{\setlength{\labelsep}{0.1in}
                       \setlength{\topsep}{0in}
                       \setlength{\partopsep}{0in}
                       \setlength{\leftmargin}{0.5in}
                       \setlength{\labelwidth}{0.5in}
                       \setlength{\listparindent}{0in}
                       \setlength{\itemindent}{0in}
                       \setlength{\itemsep}{0in}
                       }
                \item{{\em Instance:\ }}{#2}
                \item{{\em Query:\ }}{#3}
                \end{list}
\end{list}
\medskip
}

%%%%%%%%%%%%%%%%%%%%%%%%%%%%%%%%%%%%%%%%%%%%%%%%%%%%%%%%%%%%%%%%%%%%%%%%%%%%%%%%%%%
%%%%%%%%%%%%% for quizzes
%%%%%%%%%%%%%%%%%%%%%%%%%%%%%%%%%%%%%%%%%%%%%%%%%%%%%%%%%%%%%%%%%%%%%%%%%%%%%%%%%%%

\newcommand{\quizheader}{ {\large NAME: \hskip 3in SID:\hfill}
                                \newline\lineacross \medskip }


%%%%%%%%%%%%%%%%%%%%%%%%%%%%%%%%%%%%%%%%%%%%%%%%%%%%%%%%%%%%%%%%%%%%%%%%%%%%%%%%%%%
%%%%%%%%%%%%% for final
%%%%%%%%%%%%%%%%%%%%%%%%%%%%%%%%%%%%%%%%%%%%%%%%%%%%%%%%%%%%%%%%%%%%%%%%%%%%%%%%%%%

\newcommand{\namespace}{\noindent{\Large NAME: \hfill SID:\hskip 1.5in\ }\\\medskip\noindent\mbox{}\hrulefill\mbox{}}


%%%%%%%%%%%%%%%%%%%%%%%%%%%%%%%%%%%%%%%%%%%%%%%%%%%%%%%%%%%%%%%%%%%%%%%%%%%%%%%%%%%
%%%%%%%%%%%%% for notes
%%%%%%%%%%%%%%%%%%%%%%%%%%%%%%%%%%%%%%%%%%%%%%%%%%%%%%%%%%%%%%%%%%%%%%%%%%%%%%%%%%%


\newtheorem{theorem}{Theorem}[section]
\newtheorem{definition}[theorem]{Definition}
\newtheorem{corollary}[theorem]{Corollary}
\newtheorem{lemma}[theorem]{Lemma}
\newtheorem{fact}[theorem]{Fact}
\newtheorem{claim}[theorem]{Claim}

\newenvironment{proof}{{\it Proof:\/}}{$\Box$\vskip 0.1in}


                                                                                                   
\usepackage[margin=0.75in]{geometry}

\pagestyle{empty}

\begin{document}

\student{Zachary \c Sells}{861013217}

\centerline{\large \bf CS141 ASSIGNMENT 2}
\centerline{due Thursday, October 17}

\vskip 0.2in
\noindent{\bf Individual assignment:} Problems 1 and 2.

\noindent{\bf Group assignment:} Problems 1,2 and 3.

\vskip 0.2in

%%%%%%%%%%%%%%%%%%%%%%%%%%%%

\begin{problem} 
Each row of the table below contains parameters of the
RSA crypto-system: $p$, $q$, $n$, $e$, and $d$, with some
of them missing. The next-to-last column contains a 
message $x$ and the last column the corresponding ciphertext $y$.
If the present parameters are correct,
fill in the remaining entries in the row.
Otherwise, explain why the parameters are not correct. Show your work.

\vskip 0.2in

\noindent
\begin{tabular}{|c|c|c|c|c||c|c|}
\hline
\ \ $p$\ \ &\ \ $q$\ \ &\ \ $n$\ \ &\ \ $e$\ \ &\ \ $d$\ \ &\ \ $x$\ \ &\ \ $y$\ \ 
\\ \hline
5 & 13  & 65 & 7  & 7 & 2  & 63 \\
\hline
%
*Error 1*  & 11  & 121 & *Error 1*  & 5  & *Error 1* & 2 \\
\hline
%
11 & 13  & 143 & 17  & 113 & 24 & 7 \\
\hline
%
7 & 11  & 77 & 5  & *Error 2* & 3  & *Error 2* \\
\hline
%
19 & 7  & 133 & 5  & 65 & 2  & 32 \\
\hline
\end{tabular}


\end{problem}

\begin{solution}

Row 1 -
\\*
To find $q$, we need to compute $q = n/p$ which gives us 13
\\*\\*
To find $d$, we need to compute $d = e^{-1} mod (p-1)(q-1)$

\xxx$d = 7^{-1} mod 48$
\\*
\xxx By listing the multiples of $7$ and of $48$ we can see which multiple of $7$ is one less than a multiple of $48$
\\*
\xxx $7$ multiples - $7, 14, 21, 28, 35, 42, 49, ...$
\\*
\xxx $48$ multiples - $48, 96, ...$
\\*
\xxx We can see that $7*7 = (49 * 1) + 1$ which gives us our answer, $d=7$
\\*
\xxx To encrypt $x$, we need to use the formula, $y = x^{d} mod n$, or $y = 2^7 mod(65)$
\\*
\xxx Simplifying, we get $128 mod(65)$.
\\*
\xxx $128 / 65 = 1$ with remainder $63$. So $128 mod(65) = 63 = y$
\\*
\\* \\* \\*

*Error 1* - This configuration of RSA will not work because p and q cannot be equal
\\*
$p = n/q = 121/11 = 11 = q$
\\*\\*
*Error 2* - This configuration of RSA will not work because e must be relatively prime both (q-1) and (p-1)
\\*
$e = 5$ and $(q-1) = 10$. $5$ and $10$ share a common factor of $2$. Therefor e is not relatively prime to (q-1)
\\*\\*
                
           

\end{solution}


%%%%%%%%%%%%%%%%%%%%%%%%%%%%

\newcommand{\vectwo}[2]{{\brackd{\begin{array}{c} #1 \\ #2 \end{array}}}}
\newcommand{\mattwo}[4]{{\brackd{\begin{array}{cc} #1 & #2 \\ #3 & #4\end{array}}}}


\begin{problem}
For an $n$ that is a power of $2$, the $n\times n$ Weirdo matrix $W_n$ is
defined as follows. For $n=1$, $W_1 = [1]$. For $n > 1$, $W_n$ is defined
inductively by
%
\begin{eqnarray*}
		W_n &=& \mattwo{ W_{n/2} }{ -W_{n/2} } { I_{n/2} } {W_{n/2} },
\end{eqnarray*}
%
where $I_k$ denotes the $k\times k$ identity matrix (whose diagonal entries are $1$
and all other entries $0$). For example,
%
\begin{eqnarray*}
	W_2 &=& \brackd{\begin{array}{cc}
					1 & -1 \\
					1 & 1
				\end{array}
				}
\quad\quad
	W_4 \;=\; \brackd{\begin{array}{cccc}
					1 & -1 & -1 & 1 \\
					1 & 1  & -1 & -1 \\
					1 & 0 & 1 & -1 \\
					0 & 1 & 1 &  1
				\end{array}
				}
\quad\quad
	W_8 \;=\; \brackd{\begin{array}{cccccccc}
	1 & -1 & -1 & 1 & -1 & 1 & 1 & -1 \\
	1 & 1  & -1 & -1 & -1 & -1 & 1 & 1 \\
	1 & 0 & 1 & -1   & -1 & 0 & -1 & 1 \\
	0 & 1 & 1 &  1  & 0 & -1 & -1 & -1 \\
	1 & 0 & 0 & 0 & 1 & -1 & -1 & 1 \\
	0 & 1 & 0 & 0 & 1 & 1  & -1 & -1 \\
	0 & 0 & 1 & 0 & 1 & 0 & 1 & -1   \\
	0 & 0 & 0 & 1 & 0 & 1 & 1 &  1
	\end{array}
	}
\end{eqnarray*}
%
Give an efficient algorithm that for a vector $\barx$ of length $n$ (where $n$
is a power
of $2$) computes the product $W_n\cdot \barx$. Your algorithm must run in time
$O(n\log n)$. (Hint: use divide-and-conquer, taking advantage of the recursive
definition of $W_n$.)
\end{problem}

\begin{solution}

If we write out explicitly, the solutions for $W_2X_2, W_4X_4, and$ $W_8,X_8$ it becomes apparent that each quadrant of the matrix can be expressed by the previous term in the series of matrices. 
\\*
Let's label our matrix as follows:

\begin{eqnarray*}
	W_N &=& \brackd{\begin{array}{cc}
					A & B \\
					C & D
				\end{array}
				}
\end{eqnarray*}
\\*\\*
Now we can construct the following algorithm to compute $W_n * X_n$
\\*\\*

\xxx def mult($X_n$, $W_n$):


	\xxx$if(n = 1)$
		return $X_n$
	
	\xxx$quadrant_A = mult(X_{n/2}, W_{n/2}))$
	
	\xxx$quadrant_B = -1 * mult(X_{n/2 - n}, W_{n/2})$
	
	\xxx$quadrant_C = [X_n]$
	
	\xxx$quadrant_D = quadrant_A$
	
\begin{eqnarray*}
	return
	\brackd{\begin{array}{cc}
					A + B \\
					C + D
				\end{array}
				}
\end{eqnarray*}

From this, we can get the reccurance equation, which is: 
\\*\\*
$T(n) = 2T(n/2) + O(n)$
\\*\\*

Using master's theorem, with condition, $k = 0$, we can obtain the asymptotic value of this recurrance function:
\\*\\*
$\theta(n log(n))$
\\*

\end{solution}


%%%%%%%%%%%%%%%%%%%%%%%%%%%%

\begin{problem} 
Prof. Goofy has been trying to speed-up the divide-and-conquer integer multiplication algorithm
as follows: Given two numbers $x,y$ with $n$ bits each (you can assume that $n$ is
a power of $4$), he wants to:
%
\begin{description}
	\item{(i)} divide each into four equal-length pieces (instead of two pieces as before),
and 
	\item{(ii)} express the product $x\cdot y$ using some number $p$ of
multiplications of these $n/4$-bit pieces, and some additions, subtractions or shifts.
%
\end{description}
%
How small does $p$ need to be for Prof. Goofy's idea to give a faster
algorithm than the $O(n^{\log_2{3}})$-time algorithm
covered in class? Justify your answer. 
\end{problem}

%%%%%%%%%%%%%%%%%%%%%%%%%%%%

%\vskip 0.1in
%\paragraph{Submission.}
%The homework needs to be submitted in hardcopy in class on Friday, October 27. 
%
\end{document}

